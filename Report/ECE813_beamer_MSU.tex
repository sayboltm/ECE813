%\documentclass[12pt]{article}
%\usepackage{amsmath}
%\usepackage{graphicx}
%\usepackage{float}
%\usepackage[font={small,it}]{caption}
%\pagenumbering{arabic}

%\graphicspath{{./img/}{~/GitHub/sayboltm/PHY480/Project3/Report/img/}}
%\graphicspath{{./img/}}

% This is a comment

%%%%% Attempt at beamer
%\documentclass[t]{beamer} % [t] top align slides globally
%\documentclass[xcolor=dvipsnames]{beamer} % [xcolor] thing to adjust color for msu theme
\documentclass[t, xcolor=dvipsnames]{beamer} % Custom color and top centered

%%%%%%% Theme experimentation
%\usetheme{Boadilla}
%\usetheme{Structure}
\usetheme{default}
\usepackage[T1]{fontenc} % Added later for < > signs
\usebackgroundtemplate{\includegraphics[width=\paperwidth]{./MSUbak.PNG}}

%%% Color config
%\setbeamercolor{normal text}{fg=black, bg=white}

\makeatletter
%\definecolor{beamer@blendedblue}{rgb}{0.5,0.5,0.3} % default
%\definecolor{beamer@blendedblue}{rgb}{1,1,1} % this is title/bullets
\definecolor{beamer@blendedblue}{rgb}{0,0,0} % black
%\definecolor{beamer@blendedblue}{rgb}{23/69,69/69,59/69} % Attempt green
%\definecolor{beamer@blendedblue}{rgb}{.333,1,.855} % Attempt green 2 
% MSU COLOR: {rgb}{hsl}{23,69,59}{111,120,43}
\setbeamercolor{normal text}{fg=black,bg=white}
\setbeamercolor{alerted text}{fg=red}
\setbeamercolor{example text}{fg=green!50!black}

\setbeamercolor{structure}{fg=beamer@blendedblue} % Thisis the custom color insert

\setbeamercolor{background canvas}{parent=normal text}
\setbeamercolor{background}{parent=background canvas}

\setbeamercolor{palette primary}{fg=yellow,bg=yellow} 
\setbeamercolor{palette secondary}{use=structure,fg=structure.fg!100!green}
\setbeamercolor{palette tertiary}{use=structure,fg=structure.fg!100!green} 
\makeatother



\begin{document}

\begin{frame}
\title{Crazyflie Adventures: 2.0 Edition}
\subtitle{Analysis and setup}
\author{Michael Saybolt, Joe Fabbo}
\institute{Michigan State University}
%\date{\today}
\date{5/4/2017}
\titlepage
\end{frame}

\begin{frame}
	\frametitle{Outline}
	\tableofcontents
\end{frame}

\section{Introduction}
\subsection{Motivation}
\begin{frame}
	\frametitle{Introduction.Motivation}
	\begin{itemize}
		\item{Why Crazyflie?}
		\begin{itemize}
			\item Modular, open source development platform
			\item Very accurate sensors
			\item Active development community
			\item Scattered documentation, but this presentation and report aim to fix that
		\end{itemize}
	\end{itemize} %TODO: Put an image of crazyflie here
\end{frame}

\subsection{Development Background}
\begin{frame} %[t] % Can also put t here for local alignment
	\frametitle{Introduction.DevelopmentBackground}
	\begin{itemize}
		\item{Crazyflie works on Linux, Windows, and Mac}
		\item{Most developers use Linux}
		\item VM available for quick setup but native environment works better
			\begin{itemize}
				\item Virtualbox USB driver issues
				\item VM software introduces variables CF devs can't control
			\end{itemize}
		\item{This was tested on Ubuntu 16.04 LTS AMD64}
		\item Crazyflie Client 2016.4 (though new versions should work)
	\end{itemize}
\end{frame}

\section{Setup}
\subsection{Software Package Requirements}
\begin{frame}
	\frametitle{Software Package Requirments}
	\begin{itemize}
		\item Git
			\begin{itemize}
				\item To download and maintain Crazyflie programs
			\end{itemize}
		\item Python virtual environment
			\begin{itemize}
				\item Dependency management
			\begin{itemize}
				\item Anaconda virtual environment
				\item virtualenv
			\end{itemize}
			\end{itemize}
		\item Docker
			\begin{itemize}
				\item Vitual environment outside of Python
				\item Used for installation and use of Bitcraze toolbelt
				\item Ensures same compilers are used for compiling Windows binaries, potentially Android/iOS, or firmware
				\item Shouldn't be needed for development of CF Client
				\item For more information see:
				\item https://github.com/bitcraze/toolbelt
			\end{itemize}
	\end{itemize}
\end{frame}

\section{Procedure for Basic Development}
\begin{frame}
	\frametitle{Procedure for Basic Development}
	\begin{itemize}
		\item (Optional) Install Docker/toolbelt
		\item Fork/clone crazyflie-clients-python from GitHub
			\begin{semiverbatim}
			\ git clone https://github.com/bitcraze/crazyflie-clients-python
			\end{semiverbatim}
		\item Set up Python virtual environment
		\item Install necessary dependencies
		\item Set udev permissions
		\item Run Crazyflie client, you can modify the source code
	\end{itemize}
\end{frame}

\subsection{Setting up Python Virtual Environment}
\begin{frame}
	\frametitle{Setting up Python Virtual Environment}
	\begin{itemize}
		\item Most of the documentation states to use virtualenv
		\begin{itemize}
			\item Contains dependencies installed with pip, a Python package manager in an isolated environment
			\item Pip struggles to install some dependencies needed to run the CF client
		\end{itemize}
		\item Solution: use Anaconda/Miniconda
		\begin{itemize}
			\item Anaconda is a Python suite including Python, conda, a python package manager, and many useful Python packages	
			\item Conda package manager handles finnicky package installs better than pip
			\item Conda also has an implementation of virtual environments that can handle packages installed with conda as well as pip 
			\item Some preconfigured packages in Anaconda don't play nice with the required ones for cfclient
			\item Miniconda contains just Python and conda
			\item Miniconda can be used to setup 'naked' conda virtual environments and install only what is needed and works
		\end{itemize}
	\end{itemize}
\end{frame}

\begin{frame}
	\frametitle{Setting up Python Virtual environment (cont.)}
	\begin{itemize}
		\item Download and install latest Miniconda setup
		\begin{itemize}
			\item It should add an alias to the 'conda' command in your .bashrc
			\item This lets you run conda from any folder by typing 'conda'
		\end{itemize}			
		\item Create a conda virtual environment for the CF Client install and development
		\begin{itemize}
			\item Name it something useful, like 'crazyfliedev'
		\end{itemize}
			\begin{semiverbatim}
			\ conda create -n crazyfliedev
			\end{semiverbatim}
		\item Your conda empty virtual environment is now ready for dependency installation
	\end{itemize}
\end{frame}

\subsection{Install dependencies}
\begin{frame}
	\frametitle{Installing Dependencies}
	\begin{itemize}	
		\item Hard work of solving proper dependencies can be summed up
		\item .. Into a conda virtual environment configuration file!
		\item Simply clone (or view/search) this github repository:
			\begin{semiverbatim}
			\ git clone https://github.com/sayboltm/ECE813.git
			\end{semiverbatim} 
		\begin{itemize}
			\item Locate File:
			\item ECE813/environments/crazyflie.yml	
			\item Create a new conda virtual environment from file!
				\begin{semiverbatim}
				\ conda env create -f crazyflie.yml
				\end{semiverbatim}
			\item Activate it:	
				\begin{semiverbatim}
				\ source activate crazyflie
				\end{semiverbatim}
		\end{itemize}
	\end{itemize}
\end{frame}
\begin{frame}
	\frametitle{Installing Dependencies (cont.)}
	\begin{itemize}
		\item Else, (say, in case of failure to run CF client with those dependencies included in crazyflie.yml)
		\begin{itemize}
			\item Activate previously created 'naked' environment
			\begin{semiverbatim}
			\ source activate crazyfliedev
			\end{semiverbatim}
			\item Run setup file
			\begin{semiverbatim}
			\ cd /path/to/crazyflie-clients-python/
			\end{semiverbatim}
			\begin{semiverbatim}
			\ python setup.py
			\end{semiverbatim}

		\end{itemize}
		\item Install any dependencies inside your conda virtual environment
		\item Deactivate the virtual environment when not needed
			\begin{semiverbatim}
			\ source deactivate
			\end{semiverbatim}
		\item Any modification or use of the CFC will require the environment to be active
	\end{itemize}
\end{frame}

\begin{frame}
	\frametitle{Setting USB Device (udev) permissions}
	\begin{itemize}
		\item Setting udev permissions allows access to usb devices without root permissions
		\item It is bad practice to use root unless absolutely necessary
		\item Crazyflie development is not a good reason
		\item Add yourUsername to group plugdev if it is not already
		\begin{itemize}
			\item Check to see what groups you are in	
				\begin{semiverbatim}
				\ groups yourUsername
				\end{semiverbatim}
			\item Add yourself or create and add if needed
				\begin{semiverbatim}
				\ sudo groupadd plugdev
				\end{semiverbatim}
				\begin{semiverbatim}
				\ sudo usermod -a -G plugdev yourUsername
				\end{semiverbatim}
		\end{itemize}
	\end{itemize}
\end{frame}

\section{I Broke It}
\begin{frame}
	\frametitle{I broke it}
	\begin{itemize}
		\item Adding or upgrading dependencies can break others
		\item Importance in using virtual Python environments
		\item Easy to save, recreate old environment
		\item Recommended that you create a new virtual environment when upgrading
		\item Pull the latest Crazyflie client, then use pip to install any new dependencies
			\begin{semiverbatim}
			\ cd /path/to/crazyflie-clients-python/
			\end{semiverbatim}
			\begin{semiverbatim}
			\ git pull
			\end{semiverbatim}
			\begin{semiverbatim}
			\ pip install -e ./
			\end{semiverbatim}
      \end{itemize}
\end{frame}

\section{To Do}
\begin{frame}
	\frametitle{To Do}
	\begin{itemize}
		\item Fix Toolbelt setup
		\item Do some development with the CF Client
		\begin{itemize}
			\item ZMQ backend allows for external input to the CF Client without needing to understand much of the CF Client source code
			\item Source code is Python so easy to read anyways
			\item Implement supplemental reactive learning
			\item Implement local positioning project
		\end{itemize}
	\end{itemize}
\end{frame}


\section{Software Challenges}
\begin{frame}
	\frametitle{Challenges: Sounds easy right?}
	\begin{itemize}
		\item Python wrapped C libraries are finnicky
			\begin{itemize}
				\item E.g. QT 
				\item Updating QT for latest CFClient causes your Python-based IDE to segfault (underlying C implementation issue)
			\end{itemize}
		\item Different versions of Python can also create issues
			\begin{itemize}
				\item py2exe not supported on Python > 3.4
				\item QT5 requires Python > 3.4
				\item Latest CFClient requires QT5
				\item Conflict breaks environment used for other projects/classes
				\item Must fix Crazyflie dev environment and tools for other coursework, find way for them to coexist
			\end{itemize}
		\item Hence need for easy to use virtual environments
			\begin{itemize}
				\item conda virtual environments > virtualenv
			\end{itemize}
		\item Documentation is scattered, lots of trial and error
			\begin{itemize}
				\item Still recommends virtualenv
				\item Conda virtual environments are superior
				\item Conda > pip for package management
				\item Conda env handles conda and pip managed packages
			\end{itemize}
	\end{itemize}
\end{frame}


%\begin{frame}
%	\frametitle{}
%\end{frame}
%
%\begin{semiverbatim}
%\end{semiverbatim}

\end{document}
